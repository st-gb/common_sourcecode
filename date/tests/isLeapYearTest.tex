%%(c) Stefan Gebauer(Computer Science Master from TU Berlin)
%%@author Stefan Gebauer(TU Berlin matriculation number 361095)

\documentclass[]{article}

%%https://riptutorial.com/latex/example/28656/if-statements
\usepackage{ifthen}%%\ifthenelse
%%http://tex.stackexchange.com/questions/87824/how-to-do-simple-calculation-in-latex
\usepackage{intcalc}
\usepackage{amsmath}%%"\implies"

\title{"is leap year function" test}
\author{Stefan Gebauer(TU Berlin matriculation number 361095)}

%%(c) Stefan Gebauer(Computer Science Master from TU Berlin)
%%@author Stefan Gebauer(TU Berlin matriculation number 361095)

%%Include guard(see http://en.wikipedia.org/wiki/Include_guard) to prevent
%% defining(via "\newcommand") the command more than once.
\ifdefined \TUBlnThSiTenNiFiIsLeapYear
\else

%%http://de.wikipedia.org/wiki/Schaltjahr#Gregorianischer_Kalender
\newcounter{TUBlnThSiTenNiFiLeapYear}
\newcounter{TUBlnThSiTenNiFiIsLeapYearYearModFour}
\newcounter{TUBlnThSiTenNiFiIsLeapYearYearModOneHoundred}
\newcounter{TUBlnThSiTenNiFiIsLeapYearYearModFourHoundred}

\newcommand{
%%Prefix with author's TU Berlin matriculation number to avoid name clashes.
%%TUBlnThSiTenNiFi=TU Berlin THree SIx TEN NIne FIve (36 10 95)
  \TUBlnThSiTenNiFiIsLeapYear}[
 %%Parameter "#1":year to give result for if it is a leap year
  1]
{
  %%Use counters because using the result in variables (via "\(re)newcommand")
  %% did not work(?)
  \setcounter{TUBlnThSiTenNiFiIsLeapYearYearModFour}{
  %%Needs package "intcalc"->"\usepackage{intcalc}" before "\begin{document}".
    \intcalcMod{#1}{4} }
  \setcounter{TUBlnThSiTenNiFiIsLeapYearYearModOneHoundred}{
  %%Needs package "intcalc"->"\usepackage{intcalc}" before "\begin{document}".
    \intcalcMod{#1}{100} }
  \setcounter{TUBlnThSiTenNiFiIsLeapYearYearModFourHoundred}{
  %%Needs package "intcalc"->"\usepackage{intcalc}" before "\begin{document}".
    \intcalcMod{#1}{400} }
  \ifthenelse%%"if" branch of "\ifthenelse":
  {
    \theTUBlnThSiTenNiFiIsLeapYearYearModFour=0
%%http://tex.stackexchange.com/questions/136831/boolean-operators-can-i-use-and-or-in-ifthenelse-how
    \and
		%%https://texdoc.org/serve/ifthen.pdf/0 : "parentheses \( \)"
    \( \not \theTUBlnThSiTenNiFiIsLeapYearYearModOneHoundred=0 \or
      \theTUBlnThSiTenNiFiIsLeapYearYearModFourHoundred=0\)
  }
  {%%"then"(<=>is leap year) branch of "\ifthenelse":
%%http://tex.stackexchange.com/questions/37094/what-is-the-recommended-way-to-assign-a-value-to-a-variable-and-retrieve-it-for
    \setcounter{TUBlnThSiTenNiFiLeapYear}{1}
  }
  {%%"else"(<=>no leap year) branch of "\ifthenelse":
%%http://tex.stackexchange.com/questions/37094/what-is-the-recommended-way-to-assign-a-value-to-a-variable-and-retrieve-it-for
    \setcounter{TUBlnThSiTenNiFiLeapYear}{0}
  }
}

\fi%%Include guard


\newcounter{year}
  
\newcommand{\testIfLeapYear}[1]{%%#1: year
  \setcounter{year}{#1}
  \isLeapYear{\theyear}
 
  \theyear
    \ifthenelse{\theleapYear=1}%%if
      { \textbf{is} }%%then
      { is \textbf{not} }%%else
    a leap year:\newline
%%Make comment to prevent a new paragraph on an empty new line.
%%
  year(\theyear) mod 4:\theyearModFour
  \ifthenelse{
  	\theyearModFour=0}%%if
    { year mod 4 = 0 $\implies$ potential leap year}%%then
    { $\implies$ \textbf{no} leap year}%%else
  \newline
  year(\theyear) mod 100:\theyearModOneHoundred
  \ifthenelse{
  	\theyearModOneHoundred=0}%%if
    { year mod 100 = 0 $\implies$ potentially no leap year}%%then
    {}%%else
  \newline
  year(\theyear) mod 400:\theyearModFourHoundred
  \ifthenelse{
  	\theyearModFourHoundred=0}%%if
    { year mod 400 = 0 $\implies$ leap year}%%then
    {}%%else
  \newline
}

\begin{document}
\maketitle

\testIfLeapYear{1900}
\testIfLeapYear{1999}
\testIfLeapYear{2000}
\testIfLeapYear{2024}  
\end{document}