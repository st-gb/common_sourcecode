%%(c) Stefan Gebauer(TU Berlin matriculation number 361095)

%%Following source code adapted from:
%% https://tex.stackexchange.com/questions/7590/how-to-programmatically-make-tabular-rows-using-whiledo

%%Create itemization from unrented rooms:
\makeatletter
\newtoks\@UnrentedRoomsAsItems
\newcommand\addUnrentedRoomsAsItems[1]{
 \@UnrentedRoomsAsItems\expandafter{\the\@UnrentedRoomsAsItems#1} }
\newcommand*\resetUnrentedRoomsAsItems{\@UnrentedRoomsAsItems{}}
\newcommand*\printUnrentedRoomsAsItems{\the\@UnrentedRoomsAsItems}
\makeatother

%%Prefix with author's ID to avoid name clashed of same variable names
\newcommand\TUBlnThSiTenNiFiZiQm{}
\FPset\TUBlnThSiTenNiFiRoomsQmSum{0.0}
\resetUnrentedRoomsAsItems
\newcounter{UnrentedRoomsAsItemsCtr}
\newcounter{UnrentedRoomsAsItemsLoopCtr}

%%The following code creates list items for unrented rooms in German language
%% Therefore the array \TUBlnThSiTenNiFiAllRooms and the variable
%% \TUBlnThSiTenNiFiRentedRoom must be defined.
%%Can be used for shared flats amd normal flats where 1 or more rooms aren't
%% rented.
%%The items are finally printed/written with "\printUnrentedRoomsAsItems"

%\newcommand\listUnrentedRooms%[1]
%{
%\addUnrentedRoomsAsItems{\begin{itemize}}
 \loop\ifnum\theUnrentedRoomsAsItemsLoopCtr< \TUBlnThSiTenNiFiAllRoomslen\relax
   \stepcounter{UnrentedRoomsAsItemsLoopCtr}
   \addUnrentedRoomsAsItems
   {
     \stepcounter{UnrentedRoomsAsItemsCtr}  
      %\themonthDaysCntAsTabCellsCtr \ 
     \ifthenelse
     {%%"if"-branch
       \equal{\TUBlnThSiTenNiFiAllRooms[\theUnrentedRoomsAsItemsCtr]}
        {\TUBlnThSiTenNiFiRentedRoom}
     }
     {%%"then"-branch
     }
     {%%"else"-branch
       \item Zimmer \TUBlnThSiTenNiFiAllRooms[\theUnrentedRoomsAsItemsCtr] %:
        %%from https://tex.stackexchange.com/questions/369104/how-to-declare-dynamic-variable-name-in-latex
        \renewcommand\TUBlnThSiTenNiFiZiQm{
        %%The following code dynamically generates a variable name. The name is
        %% something like "Zimmer">>name<<"Qm", for example "ZimmerWestQm".
        %%This variable name (and its [floating point] value) must exist!
          \expandafter\csname Zimmer\TUBlnThSiTenNiFiAllRooms[\theUnrentedRoomsAsItemsCtr]Qm\endcsname
        }
        %%Outputs the content (should be square meters) of the variable.
        \TUBlnThSiTenNiFiZiQm
        \space m$^2$
       \FPeval\TUBlnThSiTenNiFiRoomsQmSum{
         round(\TUBlnThSiTenNiFiRoomsQmSum+\TUBlnThSiTenNiFiZiQm
          :2%%Anzahl Nachkommastellen
         )
       }
     }
   }%
\repeat