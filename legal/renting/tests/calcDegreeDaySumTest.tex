%%(c) Stefan Gebauer(Computer Science Master from TU Berlin)
%%@author Stefan Gebauer(TU Berlin matriculation number 361095)

\documentclass[]{article}

%%https://latex.org/forum/viewtopic.php?t=14930
\usepackage{titling}%%\droptitle
%%Eliminate the default vertical space on top of the first page.
\setlength{\droptitle}{-4em}

\usepackage[a4paper, left=2.5 cm, right=2 cm, top=2 cm, bottom=3 cm]{geometry}

\usepackage{longtable}%%\begin{longtable}, \end{longtable}

\usepackage{enumitem}%%"topsep" as option in "\begin{itemize}"
\newcommand\topsepDef{topsep=-10pt}

%\usepackage[utf8]{inputenc}%%for ">"
%%http://tex.stackexchange.com/questions/87824/how-to-do-simple-calculation-in-latex
\usepackage{intcalc}
\usepackage{listofitems}
\usepackage{ifthen}%%\ifthenelse
%%http://miktex.org/packages/currfile/files
%% "currfile" has 0 bytes in "MiKTeX Console"
%% https://ftp.mpi-inf.mpg.de/pub/tex/mirror/ftp.dante.de/pub/tex/systems/win32/miktex/tm/packages/currfile.tar.lzma
%% "currfile" had to be installed via "MiKTeX console"
\usepackage{currfile}
\usepackage[debug]
  {fp}%%"\FPeval"
\usepackage{xcolor}%%"\color"

%%from https://www.heise.de/tipps-tricks/LaTeX-Anfuehrungszeichen-einfuegen-so-geht-s-4351643.html
\usepackage[ngerman]{babel}%%deutsche Anführungszeichen

%%(c) Stefan Gebauer(Computer Science Master from TU Berlin)
%%@author Stefan Gebauer(TU Berlin matriculation number 361095)

%\usepackage{fp}%%"\FPeval"

\input{%%Needs "currfile" package=>"\usepackage{currfile}" before
   \currfiledir../../date/cmp2Dates.tex}
%%TUBlnThSiTenNiFiGregorianMonthDaysCount
%%(c) Stefan Gebauer(Computer Science Master from TU Berlin)
%%@author Stefan Gebauer(TU Berlin matriculation number 361095)

%%idea/adapted from:
%% http://tex.stackexchange.com/questions/443032/array-list-of-numbers
%%The following code needs package "listofitems".
%% =>see http://guides.lib.wayne.edu/latex/packages:
%% put "\usepackage{package}" before "\begin{document}"
%% in the document where _this_ file is (in-)directly included.

%%See
%% http://mirrors.ibiblio.org/CTAN/macros/generic/listofitems/listofitems-en.pdf
%% for "\setsepchar" and "\readlist"
%%Set parsing separator for "\readlist" to space(" ") character.
\setsepchar{ }%%For/of package "listofitems"

%%Number of days in each month of Gregorian calendar.
\readlist\TUBlnThSiTenNiFiGregorianMonthDaysCount{
%%@see http://en.wikipedia.org/wiki/Month#Roman_calendar
%%@see http://en.wikipedia.org/wiki/Gregorian_calendar#Description
  31 28 31 30 31 30 31 31 30 31 30 31}

%%(c) Stefan Gebauer(Computer Science Master from TU Berlin)
%%@author Stefan Gebauer(TU Berlin matriculation number 361095)

%%http://de.wikipedia.org/wiki/Schaltjahr#Gregorianischer_Kalender
\newcounter{leapYear}
\newcounter{yearModFour}
\newcounter{yearModOneHoundred}
\newcounter{yearModFourHoundred}

%% "#1": year
\newcommand{\isLeapYear}[1]{
  \setcounter{yearModFour}{
  	%%Needs package "intcalc"->"\usepackage{intcalc}"
    \intcalcMod{#1}{4} }
  \setcounter{yearModOneHoundred}{
  	%%Needs package "intcalc"->"\usepackage{intcalc}"
    \intcalcMod{#1}{100} }
  \setcounter{yearModFourHoundred}{
  	%%Needs package "intcalc"->"\usepackage{intcalc}"
    \intcalcMod{#1}{400} }
  \ifthenelse%%if:
    {
     \theyearModFour=0
%%http://tex.stackexchange.com/questions/136831/boolean-operators-can-i-use-and-or-in-ifthenelse-how
     \AND
		%%https://texdoc.org/serve/ifthen.pdf/0 : "parentheses \( \)"
    \( \not \theyearModOneHoundred=0 \OR \theyearModFourHoundred=0\)
    }
    %%then:
    {%is leap year
%%http://tex.stackexchange.com/questions/37094/what-is-the-recommended-way-to-assign-a-value-to-a-variable-and-retrieve-it-for
      \setcounter{leapYear}{1}
    }
    %%else:
    {%no leap year
%%http://tex.stackexchange.com/questions/37094/what-is-the-recommended-way-to-assign-a-value-to-a-variable-and-retrieve-it-for
      \setcounter{leapYear}{0}
    }
}
\input{\currfiledir../../date/subtract2Dates.tex}
%%\TUBlnThSiTenNiFiGerHeatingDegDayPromillePerDayTwoDecPlaces
%%(c) Stefan Gebauer(Computer Science Master from TU Berlin)
%%@author Stefan Gebauer(TU Berlin matriculation number 361095)

%%see %%http://en.wikipedia.org/wiki/Heating_degree_day,
%% "Gradtagstabelle" ("Das Vermieter-Praxishandbuch", Kapitel "4.2.4"),
%% http://de.wikipedia.org/wiki/Gradtagzahl#Anwendung(for example version from
%%  08:09, 30. Okt. 2023‎):
%% month              | promille/month | ‰(promille)/day
%%--------------------+----------------+---------------
%% January            |       170      | 5,484
%%http://de.wikipedia.org/wiki/Gradtagzahl#Kostenrechnung(for example version
%% from 08:09, 30. Okt. 2023)‎:
%% February           |       150      | 5,357=150 ‰/mon=150 ‰/28 days
%% February(leap year)|       150      | 5,172=150 ‰/mon=150 ‰/29 days
%% March              |       130      | 4,194
%% April              |        80      | 2,667
%% May                |        40      | 1,290
%% June               |       40/3     | 0,444=(40/3) ‰/30 days=40 ‰/(3*30) days
%% July               |       40/3     | 0,430=(40/3) ‰/31 days=40 ‰/(3*31) days
%% August             |       40/3     | 0,430
%% September          |        30      | 1,000
%% October            |        80      | 2,581
%% November           |       120      | 4,000
%% December           |       160      | 5,161

%%idea/adapted from:
%% http://tex.stackexchange.com/questions/443032/array-list-of-numbers
%%The following code needs package "listofitems" included.
%% =>see http://guides.lib.wayne.edu/latex/packages:
%% put "\usepackage{package}" before "\begin{document}"
%% in the document where _this_ file is (in-)directly included.
%%See
%% http://mirrors.ibiblio.org/CTAN/macros/generic/listofitems/listofitems-en.pdf
%% for "\setsepchar" and "\readlist".
%%Set parsing separator for "\readlist" to comma(",") character. This is less
%% error-prone than space character when using line break(s)/indentation(s).
\setsepchar{,}%%For/of package "listofitems"

%%German heating degree day promille per day with 2 decimal places.
\readlist\TUBlnThSiTenNiFiGerHeatingDegDayPromillePerDayTwoDecPlaces{
  5.48,5.35,4.19,2.66,1.29,0.44,0.43,0.43,1.0,2.58,4.00,5.16}

%%German heating degree day promille per month.
\readlist\TUBlnThSiTenNiFiGerHeatingDegDayPromillePerMon{
  170 150 150 130 80 40 40/3 40/3 40/3 30 80 120 160}
%%(c) Stefan Gebauer(Computer Science Master from TU Berlin)
%%@author Stefan Gebauer(TU Berlin matriculation number 361095)

%%Include guard(see http://en.wikipedia.org/wiki/Include_guard) to prevent
%% defining(via "\newcommand") a command more than once.
\ifdefined \TUBlnThSiTenNiFiLog
\else

%%Logs to (La)Tex output(for example visible inside "TeXstudio"'s "Log" window).
\newcommand{
%%Prefix with author's TU Berlin matriculation number to avoid name clashes.
%%TUBlnThSiTenNiFi=TU Berlin THree SIx TEN NIne FIve (36 10 95)
  \TUBlnThSiTenNiFiLog}[1]{
  \wlog{#1}
  }

\fi%%Include guard%%"\TUBlnThSiTenNiFiLog{}"

%\usepackage{fp}%%\FPeval
%%https://riptutorial.com/latex/example/28656/if-statements
%\usepackage{ifthen}%%\ifthenelse
%\usepackage{listofitems}

\readlist\EnglishMonthNames{"Jan", "Feb"}

\newcounter{TUBlnThSiTenNiFiCalcDegreeDaySumCurrYear}
\newcounter{TUBlnThSiTenNiFiCalcDegreeDaySumCurrMon}
\newcounter{TUBlnThSiTenNiFiCalcDegreeDaySumCurrDay}

%%http://tex.stackexchange.com/questions/74707/how-to-concatenate-strings-into-a-single-command

%%https://tex.stackexchange.com/questions/34697/variable-name-newcommand-with-parameters-within-another-newcommand

%% TU Berlin THree SIx TEN NIne FIve (361095)
\def\TUBlnThSiTenNiFiPrfx{TUBlnThSiTenNiFi}
%\def\TUBlnThSiTenNiFi#1#2{\TUBlnThSiTenNiFiPrfx#1}
%\newcommand\TUBlnThSiTenNiFi[1]{%\textbackslash \TUBlnThSiTenNiFiPrfx#1
%	\@namedef{\TUBlnThSiTenNiFiPrfx#1}
%}
\newcommand{\TUBlnThSiTenNiFiDef}[2]{%
	\expandafter\newcommand\csname TUBlnThSiTenNiFi#1\endcsname{#2}%
	\expandafter\newcommand\csname #1\endcsname[1]{\expandafter\renewcommand\csname TUBlnThSiTenNiFi#1\endcsname{##1}}%
}
%\newcommand{\TUBlnThSiTenNiFiDef}[1]{%
%	\long\@namedef{#1}##1{\@namedef{TUBlnThSiTenNiFiDef#1}{##1}}}

%\def\TUBlnThSiTenNiFiCtr#1#2{\the\TUBlnThSiTenNiFiPrfx#1}
\newcommand{\TUBlnThSiTenNiFiCtr}[2]{\@namedef{\TUBlnThSiTenNiFiPrfx#1}{#2} }

\newcounter{TUBlnThSiTenNiFiCalcDegDaysSumNumDays}

%%Use counter because adding a value of newcommand with itself did not function.
\newcounter{TUBlnThSiTenNiFiCalcHeatngDegDayNumFullYears}
\newcommand{\TUBlnThSiTenNiFiCalcHeatngDegDayPromilleForFullYears}[
  %%The number of parameters to this function:
  %% from:year,month,day,
  %% to:year,month,day
  6%4
  ]
{
  \TUBlnThSiTenNiFiLog{TUBlnThSiTenNiFiCalcHeatngDegDayPromilleForFullYears begin}
  %%deg=DEGree:http://www.abbreviations.com/abbreviation/degree
  \TUBlnThSiTenNiFiSetVarVal{\TUBlnThSiTenNiFiCalcHeatngDegDaySumStrtYear}{#1}
  \TUBlnThSiTenNiFiSetVarVal{\TUBlnThSiTenNiFiCalcHeatngDegDaySumStrtMon}{#2}
  \TUBlnThSiTenNiFiSetVarVal{\TUBlnThSiTenNiFiCalcHeatngDegDaySumStrtDay}{#3}
  \TUBlnThSiTenNiFiSetVarVal{\TUBlnThSiTenNiFiCalcHeatngDegDaySumEndYear}{#4}
  \TUBlnThSiTenNiFiSetVarVal{\TUBlnThSiTenNiFiCalcHeatngDegDaySumEndMon}{#5}
  \TUBlnThSiTenNiFiSetVarVal{\TUBlnThSiTenNiFiCalcHeatngDegDaySumEndDay}{#6}

  \setcounter{TUBlnThSiTenNiFiCalcHeatngDegDayNumFullYears}{0}

    \ifthenelse{\TUBlnThSiTenNiFiCalcDegreeDaySumStrtDay=1 \and
      \TUBlnThSiTenNiFiCalcDegreeDaySumStrtMon=1}
    {
      \TUBlnThSiTenNiFiSetVarVal{\TUBlnThSiTenNiFiCalcHeatngDegreeDaySumFirstFullYear}{
        \TUBlnThSiTenNiFiCalcDegreeDaySumStrtYear}
    }
    {%%Time span not starting at begin of year.
      \TUBlnThSiTenNiFiSetVarVal{\TUBlnThSiTenNiFiCalcHeatngDegreeDaySumFirstFullYear}{
        \the\numexpr\TUBlnThSiTenNiFiCalcDegreeDaySumStrtYear+1\relax}
    }
    \TUBlnThSiTenNiFiLog{1st potential(depending on end date) full year from time span:
      \TUBlnThSiTenNiFiCalcHeatngDegreeDaySumFirstFullYear% \newline
      }

  %%Basic/necessary condition for at least 1 full year.
  \ifthenelse{\TUBlnThSiTenNiFiCalcDegreeDaySumEndYear >
    \TUBlnThSiTenNiFiCalcDegreeDaySumStrtYear} %\or
  {
    \ifthenelse{\TUBlnThSiTenNiFiCalcDegreeDaySumEndDay=1 \and
      \TUBlnThSiTenNiFiCalcDegreeDaySumEndMon=1}
    {
      \TUBlnThSiTenNiFiSetVarVal{\TUBlnThSiTenNiFiCalcHeatngDegreeDaySumLastFullYear}{
        \TUBlnThSiTenNiFiCalcDegreeDaySumEndYear}
    }
    {
      \TUBlnThSiTenNiFiSetVarVal{\TUBlnThSiTenNiFiCalcHeatngDegreeDaySumLastFullYear}{
        \the\numexpr\TUBlnThSiTenNiFiCalcDegreeDaySumEndYear-1\relax}
    }
    \TUBlnThSiTenNiFiLog{last potential full year from time span:
      \TUBlnThSiTenNiFiCalcHeatngDegreeDaySumLastFullYear}
  }
  {%%end year of time span <= start year of time span
    %\TUBlnThSiTenNiFiSetVarVal{\TUBlnThSiTenNiFiCalcHeatngDegDayNumFullYears}{0}
   \TUBlnThSiTenNiFiSetVarVal{\TUBlnThSiTenNiFiCalcHeatngDegreeDaySumFirstFullYear}
     {\TUBlnThSiTenNiFiCalcDegreeDaySumStrtYear}
   \TUBlnThSiTenNiFiSetVarVal{\TUBlnThSiTenNiFiCalcHeatngDegreeDaySumLastFullYear}
     {\the\numexpr\TUBlnThSiTenNiFiCalcDegreeDaySumEndYear
       %%Subtract 1 because "1" is added afterwards.
       -1
       \relax}
    \setcounter{TUBlnThSiTenNiFiCalcHeatngDegDayNumFullYears}{0}
  }
%  \TUBlnThSiTenNiFiCalcHeatngDegreeDaySumFirstFullYear \TUBlnThSiTenNiFiCalcHeatngDegreeDaySumLastFullYear
  %%"not >" <=> "<="
  \ifthenelse{\not \TUBlnThSiTenNiFiCalcHeatngDegreeDaySumFirstFullYear >
    \TUBlnThSiTenNiFiCalcHeatngDegreeDaySumLastFullYear}
  {
    %\TUBlnThSiTenNiFiSetVarVal{\TUBlnThSiTenNiFiCalcHeatngDegreeDaySumNumFullYears}
    \setcounter{TUBlnThSiTenNiFiCalcHeatngDegDayNumFullYears}
      %%If 1 year then add 1 because for example 2005-2005+1=1
      {\the\numexpr\TUBlnThSiTenNiFiCalcHeatngDegreeDaySumLastFullYear-\TUBlnThSiTenNiFiCalcHeatngDegreeDaySumFirstFullYear+1\relax}
  }
  {}
  \ifthenelse{\TUBlnThSiTenNiFiCalcHeatngDegreeDaySumFirstFullYear >
    \TUBlnThSiTenNiFiCalcHeatngDegreeDaySumLastFullYear}
  {
%    \TUBlnThSiTenNiFiSetVarVal{\TUBlnThSiTenNiFiCalcHeatngDegDayNumFullYears}{0}
    \setcounter{TUBlnThSiTenNiFiCalcHeatngDegDayNumFullYears}{0}
  }
  {}

  \TUBlnThSiTenNiFiSetVarVal{\TUBlnThSiTenNiFiCalcDegDaysYearsInNumDays}{\the\numexpr\value{TUBlnThSiTenNiFiCalcHeatngDegDayNumFullYears}*365\relax}
  %%TODO regard leap years(366 days)
  \setcounter{TUBlnThSiTenNiFiCalcDegDaysSumNumDays}
    {\the\numexpr\value{TUBlnThSiTenNiFiCalcDegDaysSumNumDays}+\TUBlnThSiTenNiFiCalcDegDaysYearsInNumDays}

  \TUBlnThSiTenNiFiLog{TUBlnThSiTenNiFiCalcHeatngDegDayPromilleForFullYears end--
    \# of days of time span till now:%\value{TUBlnThSiTenNiFiCalcDegDaysSumNumDays}
  }
}

%%@brief Should be used by first and last days of time span and 
\newcommand{\TUBlnThSiTenNiFiGetGerHeatngDegDayPromille}[2]
{
  \TUBlnThSiTenNiFiSetVarVal{\TUBlnThSiTenNiFiGetGerHeatngDegDayPromilleMonth}
    {#1}
  \TUBlnThSiTenNiFiSetVarVal{\TUBlnThSiTenNiFiGetGerHeatngDegDayPromilleYear}
    {#2}

  \TUBlnThSiTenNiFiIsLeapYear{\TUBlnThSiTenNiFiGetGerHeatngDegDayPromilleYear}

  \ifthenelse
  %%If the first year of the time span is a leap year...
  {\theTUBlnThSiTenNiFiLeapYear=1 \and
    %% ...and month is February.
    \TUBlnThSiTenNiFiGetGerHeatngDegDayPromilleMonth=2}
  {
    \TUBlnThSiTenNiFiSetVarVal
      {\TUBlnThSiTenNiFiGetGerHeatngDegDayPromilleMonthDaysCount}{29}
  }
  {%%else: not:leap year and February
    \TUBlnThSiTenNiFiSetVarVal{\TUBlnThSiTenNiFiGetGerHeatngDegDayPromilleMonthDaysCount}
      {\TUBlnThSiTenNiFiGregorianMonthDaysCount[
        \TUBlnThSiTenNiFiGetGerHeatngDegDayPromilleMonth]}
    \TUBlnThSiTenNiFiLog{
     %Not:leap year AND February
     %=
     leap year and not February or
     no leap year and February
     :
      \TUBlnThSiTenNiFiGetGerHeatngDegDayPromilleMonthDaysCount days}
  }
%  \TUBlnThSiTenNiFiLog{Number of days for
%    \TUBlnThSiTenNiFiGetGerHeatngDegDayPromilleMonth. month:
%    \TUBlnThSiTenNiFiGetGerHeatngDegDayPromilleMonthDaysCount}

%  \TUBlnThSiTenNiFiSetVarVal{\TUBlnThSiTenNiFiGetGerHeatngDegDayPromillePerMonth}
%    {
%      \TUBlnThSiTenNiFiGerHeatingDegDayPromillePerMon[
%	    \TUBlnThSiTenNiFiGetGerHeatngDegDayPromilleMonth]}
  %%http://mirrors.ibiblio.org/CTAN/macros/generic/listofitems/listofitems-en.pdf
  %% chapter "2 Read a Simple List", "Assign an item to a macro"
  %%Use "listofitems"'s "itemtomacro" to get value from array. Assigning
  %% (indirectly) with "newcommand" did not work.
  \itemtomacro\TUBlnThSiTenNiFiGerHeatingDegDayPromillePerMon
    [\TUBlnThSiTenNiFiGetGerHeatngDegDayPromilleMonth]
    \TUBlnThSiTenNiFiGetGerHeatngDegDayPromillePerMonth

%  \TUBlnThSiTenNiFiLog{promille per
%    \TUBlnThSiTenNiFiGetGerHeatngDegDayPromilleMonth. month:
%    "\TUBlnThSiTenNiFiGetGerHeatngDegDayPromillePerMonth"}

  \FPeval\TUBlnThSiTenNiFiGetGerHeatngDegDayPromillePerDay
    {\TUBlnThSiTenNiFiGetGerHeatngDegDayPromillePerMonth/\TUBlnThSiTenNiFiGetGerHeatngDegDayPromilleMonthDaysCount}

  \TUBlnThSiTenNiFiLog{%TUBlnThSiTenNiFiGetGerHeatngDegDayPromille end--
    promille per day=%
    \TUBlnThSiTenNiFiGetGerHeatngDegDayPromillePerMonth promille for
    \TUBlnThSiTenNiFiGetGerHeatngDegDayPromilleMonth. month/%
    \TUBlnThSiTenNiFiGetGerHeatngDegDayPromilleMonthDaysCount days for
    \TUBlnThSiTenNiFiGetGerHeatngDegDayPromilleMonth. month=%
    \TUBlnThSiTenNiFiGetGerHeatngDegDayPromillePerDay}
}


%%Define counter outside of function to not define it multiple times if calling this function more than once.
\newcounter{TUBlnThSiTenNiFiCalcHeatngDegDayPromilleForFullMonthsCurrMon}

\newcommand{\TUBlnThSiTenNiFiCalcHeatngDegDayPromilleForFullMonths}[
  %% start month end month
  2]
{
  \TUBlnThSiTenNiFiSetVarVal{\TUBlnThSiTenNiFiCalcHeatngDegDayPromilleForFullMonthsStartMon}{#1}
  \TUBlnThSiTenNiFiSetVarVal{\TUBlnThSiTenNiFiCalcHeatngDegDayPromilleForFullMonthsEndMon}{#2}
  %%Reset to zero for nexr call(for last month after first month of same time span or first month after last month of time span before)
  \TUBlnThSiTenNiFiSetVarVal{\TUBlnThSiTenNiFiGerHeatngDegDayFullMonthsPromille}{0}
  %\TUBlnThSiTenNiFiSetVarVal{\TUBlnThSiTenNiFiCalcHeatngDegreeDaySumCurrMon}{1}
%%https://www.matthiaspospiech.de/blog/2008/04/13/latex-variablen-if-abfragen-und-schleifen/
  %% must use a counter??
  \setcounter{TUBlnThSiTenNiFiCalcHeatngDegDayPromilleForFullMonthsCurrMon}
    {\TUBlnThSiTenNiFiCalcHeatngDegDayPromilleForFullMonthsStartMon}

  \TUBlnThSiTenNiFiLog{TUBlnThSiTenNiFiCalcHeatngDegDayPromilleForFullMonths--current month:
    \theTUBlnThSiTenNiFiCalcHeatngDegDayPromilleForFullMonthsCurrMon\ end month:
    \TUBlnThSiTenNiFiCalcHeatngDegDayPromilleForFullMonthsEndMon \newline}

  \TUBlnThSiTenNiFiSetVarVal
    {\TUBlnThSiTenNiFiCalcHeatngDegDayPromilleForFullMonthsEndMon}
    {\the\numexpr#2+1\relax}
%  \TUBlnThSiTenNiFiLog{New end month(as `end loop` condition):
%    \TUBlnThSiTenNiFiCalcHeatngDegDayPromilleForFullMonthsEndMon}

  \FPset{TUBlnThSiTenNiFiGerHeatngDegDayFullMonthsPromille}{0}
  \whiledo{\theTUBlnThSiTenNiFiCalcHeatngDegDayPromilleForFullMonthsCurrMon <
    \TUBlnThSiTenNiFiCalcHeatngDegDayPromilleForFullMonthsEndMon}
  {  %%TODO also leap year beachten
%    \TUBlnThSiTenNiFiGetGerHeatngDegDayMonthPromille{
%      \value{TUBlnThSiTenNiFiCalcHeatngDegreeDaySumCurrMon}}

    \TUBlnThSiTenNiFiSetVarVal
      {\TUBlnThSiTenNiFiGerHeatingDegDayCurrMonthPromille}
      {\TUBlnThSiTenNiFiGerHeatingDegDayPromillePerMon[
       \value{TUBlnThSiTenNiFiCalcHeatngDegDayPromilleForFullMonthsCurrMon} ]}

    \TUBlnThSiTenNiFiSetVarVal{\TUBlnThSiTenNiFiGerHeatingDegDayCurrMonthDayCnt}
      {\TUBlnThSiTenNiFiGregorianMonthDaysCount[
      \value{TUBlnThSiTenNiFiCalcHeatngDegDayPromilleForFullMonthsCurrMon}]
      }
    \TUBlnThSiTenNiFiGregorianMonthGetNumDays
      {
      \theTUBlnThSiTenNiFiCalcDegreeDaySumCurrYear
      }
      {
      \theTUBlnThSiTenNiFiCalcHeatngDegDayPromilleForFullMonthsCurrMon
      }
    \TUBlnThSiTenNiFiLog{TUBlnThSiTenNiFiCalcHeatngDegDayPromilleForFullMonths
      --Number of days in current month:
      %\TUBlnThSiTenNiFiGerHeatingDegDayCurrMonthDayCnt
      \TUBlnThSiTenNiFiGregorianMonthNumDays
      }
%  \TUBlnThSiTenNiFiSetVarVal{\TUBlnThSiTenNiFiCalcDegDaysSumNumDays}
    \setcounter{TUBlnThSiTenNiFiCalcDegDaysSumNumDays}
      {\the\numexpr\value{TUBlnThSiTenNiFiCalcDegDaysSumNumDays}+
       %\TUBlnThSiTenNiFiGerHeatingDegDayCurrMonthDayCnt
       \TUBlnThSiTenNiFiGregorianMonthNumDays
      }

    \FPset{TUBlnThSiTenNiFiGerHeatingDegDayCurrMonthPromilleFP}
      {\TUBlnThSiTenNiFiGerHeatingDegDayCurrMonthPromille}

    \TUBlnThSiTenNiFiLog{"\TUBlnThSiTenNiFiGerHeatingDegDayCurrMonthPromille"
       heating degree promille for current
      \theTUBlnThSiTenNiFiCalcHeatngDegDayPromilleForFullMonthsCurrMon. month of year
      }

    %\FPeval\TUBlnThSiTenNiFiGerHeatingDegDayCurrMonthPromilleFP
    \FPeval\TUBlnThSiTenNiFiGerHeatngDegDayFullMonthsPromille
      {\TUBlnThSiTenNiFiGerHeatngDegDayFullMonthsPromille+
      \TUBlnThSiTenNiFiGerHeatingDegDayCurrMonthPromille
      }
       %}
%    \TUBlnThSiTenNiFiGerHeatingDegDayCurrMonthPromilleFP

    \TUBlnThSiTenNiFiLog{TUBlnThSiTenNiFiCalcHeatngDegDayPromilleForFullMonths--
      current sum:\TUBlnThSiTenNiFiGerHeatngDegDayFullMonthsPromille}

    \stepcounter{TUBlnThSiTenNiFiCalcHeatngDegDayPromilleForFullMonthsCurrMon}
  }
  \TUBlnThSiTenNiFiLog{Heating degree day promille sum for full months
    \TUBlnThSiTenNiFiCalcHeatngDegDayPromilleForFullMonthsStartMon - 
    \TUBlnThSiTenNiFiCalcHeatngDegDayPromilleForFullMonthsEndMon\ 
    of time span:\TUBlnThSiTenNiFiGerHeatngDegDayFullMonthsPromille}
}

%%TODO replace by "\TUBlnThSiTenNiFiCalcHeatngDegDayPromilleForFullMonths"
\newcommand{\TUBlnThSiTenNiFiCalcHeatngDegDayPromilleForLastMonths}[
  %%The number of parameters to this function:
  %% to:year,month,day
  3
  ]
{
%  \TUBlnThSiTenNiFiSetVarVal{\TUBlnThSiTenNiFiCalcDegreeDaySumEndYear}{#1}
%  \TUBlnThSiTenNiFiSetVarVal{\TUBlnThSiTenNiFiCalcDegreeDaySumEndMon}{#2}
%  \TUBlnThSiTenNiFiSetVarVal{\TUBlnThSiTenNiFiCalcDegreeDaySumEndDay}{#3}
  
  %\TUBlnThSiTenNiFiSetVarVal{\TUBlnThSiTenNiFiCalcHeatngDegreeDaySumCurrMon}{1}
%%https://www.matthiaspospiech.de/blog/2008/04/13/latex-variablen-if-abfragen-und-schleifen/
  %% must use a counter??
  \newcounter{TUBlnThSiTenNiFiCalcHeatngDegreeDaySumCurrMon}
  \setcounter{TUBlnThSiTenNiFiCalcHeatngDegreeDaySumCurrMon}{1}
  \TUBlnThSiTenNiFiLog{current month:\theTUBlnThSiTenNiFiCalcHeatngDegreeDaySumCurrMon end mon:
    \TUBlnThSiTenNiFiCalcDegreeDaySumEndMon \newline}

  \FPset{TUBlnThSiTenNiFiGerHeatngDegDayLastMonthsPromille}{0}
  \whiledo{\theTUBlnThSiTenNiFiCalcHeatngDegreeDaySumCurrMon <
    \TUBlnThSiTenNiFiCalcDegreeDaySumEndMon}
  {  %%TODO also leap year beachten
%    \TUBlnThSiTenNiFiGetGerHeatngDegDayMonthPromille{
%      \value{TUBlnThSiTenNiFiCalcHeatngDegreeDaySumCurrMon}}

    \TUBlnThSiTenNiFiSetVarVal
      {\TUBlnThSiTenNiFiGerHeatingDegDayCurrMonthPromille}
      {\TUBlnThSiTenNiFiGerHeatingDegDayPromillePerMon[
       \value{TUBlnThSiTenNiFiCalcHeatngDegreeDaySumCurrMon} ]}

    \FPset{TUBlnThSiTenNiFiGerHeatingDegDayCurrMonthPromilleFP}
      {\TUBlnThSiTenNiFiGerHeatingDegDayCurrMonthPromille}

    \TUBlnThSiTenNiFiLog{"\TUBlnThSiTenNiFiGerHeatingDegDayCurrMonthPromille"
       heating degree promille for current
      \theTUBlnThSiTenNiFiCalcHeatngDegreeDaySumCurrMon. month of year
      }

    \FPeval\TUBlnThSiTenNiFiGerHeatngDegDayLastMonthsPromille
      {\TUBlnThSiTenNiFiGerHeatngDegDayLastMonthsPromille+
      \TUBlnThSiTenNiFiGerHeatingDegDayCurrMonthPromille
      }

    \TUBlnThSiTenNiFiLog{TUBlnThSiTenNiFiCalcHeatngDegDayPromilleForLastMonths--
      current sum:\TUBlnThSiTenNiFiGerHeatngDegDayLastMonthsPromille}

    \stepcounter{TUBlnThSiTenNiFiCalcHeatngDegreeDaySumCurrMon}
  }
  \TUBlnThSiTenNiFiLog{heating degree day promille sum for last months of time 
    span:\TUBlnThSiTenNiFiGerHeatngDegDayLastMonthsPromille}
}

\newcommand{\TUBlnThSiTenNiFiCalcHeatngDegDayPromilleForLastDays}[
  %%The number of parameters to this function:
  %% to:year,month,day
  3
  ]
{
  \TUBlnThSiTenNiFiSetVarVal{\TUBlnThSiTenNiFiCalcDegreeDaySumEndMon}
    {#2}%%2nd parameter of this function
  \TUBlnThSiTenNiFiSetVarVal{\TUBlnThSiTenNiFiCalcDegreeDaySumEndDay}
    {#3}%%3rd parameter of this function

  \ifthenelse{\TUBlnThSiTenNiFiCalcDegreeDaySumStrtMon=
    \TUBlnThSiTenNiFiCalcDegreeDaySumEndMon \and
    \TUBlnThSiTenNiFiCalcDegreeDaySumStrtYear=
    \TUBlnThSiTenNiFiCalcDegreeDaySumEndYear
    }%%if-block
    {
  \FPset\TUBlnThSiTenNiFiCalcHeatngDegreeDayLastDaysPromille
    {0.0}
  \FPset\TUBlnThSiTenNiFiGerHeatingDegDayCurrMonthPromilleDayTwoDecPlaces{0.00}
    }%%then-block
    {%%else-block
  %\TUBlnThSiTenNiFiSetVarVal{\TUBlnThSiTenNiFiCalcDegDaysSumNumDays}
  \setcounter{TUBlnThSiTenNiFiCalcDegDaysSumNumDays}
    {\the\numexpr\value{TUBlnThSiTenNiFiCalcDegDaysSumNumDays}+\TUBlnThSiTenNiFiCalcDegreeDaySumEndDay\relax}

  \TUBlnThSiTenNiFiLog{TUBlnThSiTenNiFiCalcHeatngDegDayPromilleForLastDays  begin--month:
    \TUBlnThSiTenNiFiCalcDegreeDaySumEndMon}
  
  \TUBlnThSiTenNiFiGetGerHeatngDegDayPromille{
    \TUBlnThSiTenNiFiCalcDegreeDaySumEndMon}
    {\TUBlnThSiTenNiFiCalcDegreeDaySumEndYear}

  \TUBlnThSiTenNiFiSetVarVal
    {\TUBlnThSiTenNiFiGerHeatingDegDayCurrMonthPromilleDayTwoDecPlaces}
    {\TUBlnThSiTenNiFiGerHeatingDegDayPromillePerDayTwoDecPlaces[
      \TUBlnThSiTenNiFiCalcDegreeDaySumEndMon]}
  \TUBlnThSiTenNiFiLog{TUBlnThSiTenNiFiCalcHeatngDegDayPromilleForLastDays
    promille per day:
    `\TUBlnThSiTenNiFiGerHeatingDegDayCurrMonthPromilleDayTwoDecPlaces`
    }

  %%http://mirrors.ibiblio.org/CTAN/macros/generic/listofitems/listofitems-en.pdf
  %% chapter "2 Read a Simple List", "Assign an item to a macro"
  %%Use "listofitems"'s "itemtomacro" to get value from array. Assigning
  %% (indirectly) with "newcommand" did not work.
  \itemtomacro\TUBlnThSiTenNiFiGerHeatingDegDayPromillePerDayTwoDecPlaces
    [\TUBlnThSiTenNiFiCalcDegreeDaySumEndMon]
    \TUBlnThSiTenNiFiGerHeatingDegDayCurrMonthPromilleDayTwoDecPlacesFP
  
  %\TUBlnThSiTenNiFiLog{heating degree day for current month: "\TUBlnThSiTenNiFiGerHeatingDegDayCurrMonthPromilleDayTwoDecPlacesFP"}

  \FPeval\TUBlnThSiTenNiFiCalcHeatngDegreeDayLastDaysPromille
    {\TUBlnThSiTenNiFiGerHeatingDegDayCurrMonthPromilleDayTwoDecPlacesFP*
     \TUBlnThSiTenNiFiCalcDegreeDaySumEndDay
%     %%Promille
%     /1000.0
    }
  }

  \TUBlnThSiTenNiFiLog{"
    \TUBlnThSiTenNiFiGerHeatingDegDayCurrMonthPromilleDayTwoDecPlaces"(heating
    degree day promille for month)*%%Use comment to avoid space character in
    %% output.
    "\TUBlnThSiTenNiFiCalcDegreeDaySumEndDay"(number of days)=%%Use comment to
    %% avoid space character in output.
    "\TUBlnThSiTenNiFiCalcHeatngDegreeDayLastDaysPromille`
    }
}

%%frist days of of first (non-full) month of time span
\newcommand{\CalcHeatngDegDays}[0]
{
  \TUBlnThSiTenNiFiLog{CalcHeatngDegDays begin}
  %%Only multiply degree days if the first month of the time span is not a full
  %% month.
  %%The first month of the time span is not a full month if either(logical "or"):
  %%-the first day of the first month of the time span is > 1
  %%-the last day of the time span is < number of days of the same month
  %% AND
  %% the first month of the time span is the same as the last month of the time
  %% span(<=>the first year of the time span is the same as the last year)
  \ifthenelse
  {%%"if" branch of "\ifthenelse"
    %\theTUBlnThSiTenNiFiSubtractTwoDatesYearDiff > 0
  	%%<=>1st year is no full year
  	%\and
  	%%First day of time span is not the first day of month.
  	\theTUBlnThSiTenNiFiCalcDegreeDaySumCurrDay>1
     \or
      \(
      \TUBlnThSiTenNiFiCalcDegreeDaySumEndDay<\TUBlnThSiTenNiFiNumFirstMonthDays
      \and \TUBlnThSiTenNiFiCalcDegreeDaySumStrtMon=\TUBlnThSiTenNiFiCalcDegreeDaySumStrtMon
      \and \TUBlnThSiTenNiFiCalcDegreeDaySumStrtYear=\TUBlnThSiTenNiFiCalcDegreeDaySumStrtYear
      \)
    %\theTUBlnThSiTenNiFiCalcDegreeDaySumCurrMon>1
  }
  %%First day of the time span is not the first day of month=>use heating degree
  %% promille based on single days.
  {%%"then" branch of "\ifthenelse"
    %TUBlnThSiTenNiFiSetVarVal{\currYear}{\currYear+1}
    \TUBlnThSiTenNiFiLog{1st month of time span is no full %year
      month:1st day %>
      greater 1 or:last day of time span < \# days of last month and 
      last month=first month and last year=first year
      % or first day\<\# days in month(\TUBlnThSiTenNiFiNumFirstMonthDays)
    }

    \TUBlnThSiTenNiFiGetGerHeatngDegDayPromille
      {%\theTUBlnThSiTenNiFiCalcDegreeDaySumCurrMon
       \TUBlnThSiTenNiFiCalcDegreeDaySumStrtMon}
      {\TUBlnThSiTenNiFiCalcDegreeDaySumStrtYear}

    \FPset{TUBlnThSiTenNiFiGetGerHeatngDegDayPromilleFirstDays}
      {\TUBlnThSiTenNiFiGetGerHeatngDegDayPromillePerDay}

    \TUBlnThSiTenNiFiLog{Number of days for %current
      first (%
      %\theTUBlnThSiTenNiFiCalcDegreeDaySumCurrMon
      \TUBlnThSiTenNiFiCalcDegreeDaySumStrtMon.)month of time span:%
      %\TUBlnThSiTenNiFiCalcDegreeDaySumCurrMonDaysCount
      \TUBlnThSiTenNiFiGetGerHeatngDegDayPromilleMonthDaysCount}

    \ifthenelse%%If time span/duration < 1 month:
    %%First month and last month of time span is the same...
      {\theTUBlnThSiTenNiFiCalcDegreeDaySumCurrMon=
           \TUBlnThSiTenNiFiCalcDegreeDaySumEndMon
    %%...and first year and the last year of time span is the same.
         \and \theTUBlnThSiTenNiFiCalcDegreeDaySumCurrYear=
         \TUBlnThSiTenNiFiCalcDegreeDaySumEndYear
      }%%if
      {%%then
        \TUBlnThSiTenNiFiSetVarVal{
          \TUBlnThSiTenNiFiCalcDegreeDaySumNumDaysTillMonEnd}
            %%Without "relax" multiplying with this number results in "0".
        {%% https://tex.stackexchange.com/questions/245635/formal-syntax-rules-of-dimexpr-numexpr-glueexpr
         %% : "numexpr" calculates integers
         \the\numexpr\TUBlnThSiTenNiFiCalcDegreeDaySumEndDay-
           \theTUBlnThSiTenNiFiCalcDegreeDaySumCurrDay\relax
        }
      }
      {%%else: time span > 1 month or time span > 1 year
        \TUBlnThSiTenNiFiLog{Time span > 1 month or time span > 1 year.
          \TUBlnThSiTenNiFiGetGerHeatngDegDayPromilleMonthDaysCount - 
          \theTUBlnThSiTenNiFiCalcDegreeDaySumCurrDay }
        \TUBlnThSiTenNiFiSetVarVal{\TUBlnThSiTenNiFiCalcDegreeDaySumNumDaysTillMonEnd}
            %%Without "relax" multiplying with this number results in "0".
            {\the\numexpr%\TUBlnThSiTenNiFiCalcDegreeDaySumCurrMonDaysCount
            \TUBlnThSiTenNiFiGetGerHeatngDegDayPromilleMonthDaysCount-
              \theTUBlnThSiTenNiFiCalcDegreeDaySumCurrDay\relax}

          \ifthenelse{\theTUBlnThSiTenNiFiCalcDegreeDaySumCurrMon=12}{
            \TUBlnThSiTenNiFiLog{Current month is december.}
            \setcounter{TUBlnThSiTenNiFiCalcDegreeDaySumCurrMon}{1} }
           { \TUBlnThSiTenNiFiLog{Current month is not december.}
             \addtocounter{TUBlnThSiTenNiFiCalcDegreeDaySumCurrMon}{1} }%%else
          \TUBlnThSiTenNiFiLog{next month:
            \theTUBlnThSiTenNiFiCalcDegreeDaySumCurrMon next day:
            \theTUBlnThSiTenNiFiCalcDegreeDaySumCurrDay}
      }

  %\TUBlnThSiTenNiFiSetVarVal{\TUBlnThSiTenNiFiCalcDegDaysSumNumDays}
  \setcounter{TUBlnThSiTenNiFiCalcDegDaysSumNumDays}
    {\TUBlnThSiTenNiFiCalcDegreeDaySumNumDaysTillMonEnd}

    \TUBlnThSiTenNiFiLog{Number of days in first month of heating degree time
      span:%%
      "%\TUBlnThSiTenNiFiCalcDegreeDaySumCurrMonDaysCount
       \TUBlnThSiTenNiFiGetGerHeatngDegDayPromilleMonthDaysCount-%%
      first day of month of heating degree time span:
      `\theTUBlnThSiTenNiFiCalcDegreeDaySumCurrDay`%%
      =\TUBlnThSiTenNiFiCalcDegreeDaySumNumDaysTillMonEnd
    }

    \FPeval\TUBlnThSiTenNiFiHeatngDegFirstDaysOfFirstMonthPromille{
      \TUBlnThSiTenNiFiCalcDegreeDaySumNumDaysTillMonEnd*
      	%\TUBlnThSiTenNiFiCalcHeatingDegDaySumCurrPromillePerDayTwoDecPlaces
        \TUBlnThSiTenNiFiGetGerHeatngDegDayPromillePerDay
      	}
    \TUBlnThSiTenNiFiLog{
      Heating degree promille for %first
      last \TUBlnThSiTenNiFiCalcDegreeDaySumNumDaysTillMonEnd 
      days of 1st month of time span:
      "%\TUBlnThSiTenNiFiCalcHeatingDegDaySumCurrPromillePerDayTwoDecPlaces
      \TUBlnThSiTenNiFiGetGerHeatngDegDayPromilleFirstDays" *
      "\TUBlnThSiTenNiFiCalcDegreeDaySumNumDaysTillMonEnd"(number of days)=
      `\TUBlnThSiTenNiFiHeatngDegFirstDaysOfFirstMonthPromille`% (current TUBlnThSiTenNiFiCalcDegreeDaySumPromille)
      }
  }
  {}%%"else" branch of "\ifthenelse"
  \TUBlnThSiTenNiFiLog{CalcHeatngDegDays end}
}

%%@brief calculates the first full heating degree months if:
%% -first year of the time span is no full year
\newcommand{\TUBlnThSiTenNiFiCalcFirstHeatngDegMonths}
{
  %%Sum up the first months till end of year or till .
  \ifthenelse%%If time span/duration < 1 year.
    %%If the first or second month(is advanced if first month of the time span
    %% is no full month) of the time span is later than January...
  {\theTUBlnThSiTenNiFiCalcDegreeDaySumCurrMon>1
    %%...if the first year of the time span is the same as the last year.
	%\and \theTUBlnThSiTenNiFiCalcDegreeDaySumCurrYear=
	%\TUBlnThSiTenNiFiCalcDegreeDaySumEndYear
    %\or    
  }%%"if" branch
  {%%"then" branch

    %%TODO use \GregorianCalendarGetNumMonthDays
    \TUBlnThSiTenNiFiSetVarVal{\TUBlnThSiTenNiFiCalcHeatngDegNumEndMonthDays}
      {\TUBlnThSiTenNiFiGregorianMonthDaysCount[
        %\TUBlnThSiTenNiFiCalcDegreeDaySumEndMon
        \theTUBlnThSiTenNiFiCalcDegreeDaySumCurrMon]}

      %%If not 1 full year.
    \ifthenelse
    {%\theTUBlnThSiTenNiFiCalcDegreeDaySumCurrMon>1 \and 
     %  \TUBlnThSiTenNiFiCalcDegreeDaySumStrtDay>1
      %%Current year may have been increased if the first month of the time span
      %% is December and no full month.
      \theTUBlnThSiTenNiFiCalcDegreeDaySumCurrYear=
        \TUBlnThSiTenNiFiCalcDegreeDaySumEndYear
    }%%"if" branch of "\ifthenelse"
    {%%"then" branch of "\ifthenelse"
      \ifthenelse
      {\TUBlnThSiTenNiFiCalcDegreeDaySumEndDay<
        \TUBlnThSiTenNiFiCalcHeatngDegNumEndMonthDays
      }%%"
      {
        \TUBlnThSiTenNiFiSetVarVal{\TUBlnThSiTenNiFiCalcDegreeDaySumLastFullMonth}
          {\the\numexpr\TUBlnThSiTenNiFiCalcDegreeDaySumEndMon-1\relax}
      }
      {
        \TUBlnThSiTenNiFiSetVarVal{\TUBlnThSiTenNiFiCalcDegreeDaySumLastFullMonth}
          {\TUBlnThSiTenNiFiCalcDegreeDaySumEndMon}
      }

      \TUBlnThSiTenNiFiSetVarVal{\TUBlnThSiTenNiFiCalcDegreeDaySumNumMonthsTillEnd}
      %%Without "relax" multiplying with this number results in "0".
            %%TODO
        {\the\numexpr%\TUBlnThSiTenNiFiCalcDegreeDaySumEndMon
          \TUBlnThSiTenNiFiCalcDegreeDaySumLastFullMonth
          -\theTUBlnThSiTenNiFiCalcDegreeDaySumCurrMon+1\relax
              %1
        }
      \ifthenelse
        { \TUBlnThSiTenNiFiCalcDegreeDaySumStrtMon=
            \TUBlnThSiTenNiFiCalcDegreeDaySumEndMon}
        {
          \FPset{TUBlnThSiTenNiFiHeatngDegFirstFullMonthsPromille}{0.0}
        }%%"then" branch of "\ifthenelse"
        {
          \TUBlnThSiTenNiFiCalcHeatngDegDayPromilleForFullMonths
            {\theTUBlnThSiTenNiFiCalcDegreeDaySumCurrMon}{
              %\TUBlnThSiTenNiFiCalcDegreeDaySumEndMon
              \TUBlnThSiTenNiFiCalcDegreeDaySumLastFullMonth}
          \TUBlnThSiTenNiFiLog{Number of full months of first year of time span=
            \TUBlnThSiTenNiFiCalcDegreeDaySumEndMon-
            \theTUBlnThSiTenNiFiCalcDegreeDaySumCurrMon+1=
            \TUBlnThSiTenNiFiCalcDegreeDaySumNumMonthsTillEnd}
        \FPset{TUBlnThSiTenNiFiHeatngDegFirstFullMonthsPromille}
          {\TUBlnThSiTenNiFiGerHeatngDegDayFullMonthsPromille}
        }
    }%%Ends "then" branch of "current year=end year of time span"
%        {}
%        \ifthenelse{\theTUBlnThSiTenNiFiCalcDegreeDaySumCurrYear<
%			\TUBlnThSiTenNiFiCalcDegreeDaySumEndYear}
    {%%"else" branch: <=> current year of the time span is <> last year of the time span.
      \FPset{TUBlnThSiTenNiFiHeatngDegFirstFullMonthsPromille}{0.0}
      \TUBlnThSiTenNiFiSetVarVal{\TUBlnThSiTenNiFiCalcDegreeDaySumNumMonthsTillEnd}
      %%Without "relax" multiplying with this number results in "0".
        {\the\numexpr12-
				\theTUBlnThSiTenNiFiCalcDegreeDaySumCurrMon+1\relax}
         \TUBlnThSiTenNiFiLog{Number of full months of first year of time span=12-
           \theTUBlnThSiTenNiFiCalcDegreeDaySumCurrMon=
           \TUBlnThSiTenNiFiCalcDegreeDaySumNumMonthsTillEnd}
      \TUBlnThSiTenNiFiCalcHeatngDegDayPromilleForFullMonths
              {\theTUBlnThSiTenNiFiCalcDegreeDaySumCurrMon}{12}
      \FPset{TUBlnThSiTenNiFiHeatngDegFirstFullMonthsPromille}
        {\TUBlnThSiTenNiFiGerHeatngDegDayFullMonthsPromille}

    \setcounter{TUBlnThSiTenNiFiCalcDegreeDaySumCurrYear}
      {\the\numexpr\theTUBlnThSiTenNiFiCalcDegreeDaySumCurrYear+1\relax}
    }

%    \FPset{TUBlnThSiTenNiFiHeatngDegFirstFullMonthsPromille}
%      {\TUBlnThSiTenNiFiGerHeatngDegDayFullMonthsPromille}
% {\TUBlnThSiTenNiFiCalcDegreeDaySumCurrMonDaysCount-\theTUBlnThSiTenNiFiCalcDegreeDaySumCurrDay}      
  }%%"then" branch of: "current month is after January"
  {
    \FPset{TUBlnThSiTenNiFiHeatngDegFirstFullMonthsPromille}{0.0}
  }%%"else" branch
}

%%@brief Calculates German heating degree day promille sum for time span between
%% 2 dates.
%%Calculation is faster if full years and full months are added than if every day is processed.
\newcommand{\TUBlnThSiTenNiFiCalcHeatngDegDaySum}[
%%Parameter #1-#3: from:year,month,day
%%Parameter #3-#6: to:year,month,day
%%Parameter #7:expected heating degree sum
  7
  ]
{
  \TUBlnThSiTenNiFiLog{TUBlnThSiTenNiFiCalcHeatngDegDaySum begin}
  \TUBlnThSiTenNiFiSetVarVal{\TUBlnThSiTenNiFiCalcDegreeDaySumStrtYear}{#1}
  \TUBlnThSiTenNiFiSetVarVal{\TUBlnThSiTenNiFiCalcDegreeDaySumStrtMon}{#2}
  \TUBlnThSiTenNiFiSetVarVal{\TUBlnThSiTenNiFiCalcDegreeDaySumStrtDay}{#3}

  \setcounter{TUBlnThSiTenNiFiCalcDegreeDaySumCurrYear}{#1}
  %\setcounter{currMon}{#2}
  \setcounter{TUBlnThSiTenNiFiCalcDegreeDaySumCurrMon}{#2}
  \setcounter{TUBlnThSiTenNiFiCalcDegreeDaySumCurrDay}{#3}
  %%Can't use "\end" as prefix (else errors) because this is a keyword in LaTex.
  \TUBlnThSiTenNiFiSetVarVal{\TUBlnThSiTenNiFiCalcDegreeDaySumEndYear}{#4}
  \TUBlnThSiTenNiFiSetVarVal{\TUBlnThSiTenNiFiCalcDegreeDaySumEndMon}{#5}
  \TUBlnThSiTenNiFiSetVarVal{\TUBlnThSiTenNiFiCalcDegreeDaySumEndDay}{#6}

  \TUBlnThSiTenNiFiSetVarVal{\TUBlnThSiTenNiFiCalcDegDaySumExpctdRslt}{#7}

  %%Used to get the number of days for _exactly_ calculating the 
  %% rental costs.
  %\TUBlnThSiTenNiFiSetVarVal{\TUBlnThSiTenNiFiCalcDegDaysSumNumDays}{0}
  \setcounter{TUBlnThSiTenNiFiCalcDegDaysSumNumDays}{0}
  
  \subtractTwoDates{\TUBlnThSiTenNiFiCalcDegreeDaySumEndDay}
   {\TUBlnThSiTenNiFiCalcDegreeDaySumEndMon}
   {\TUBlnThSiTenNiFiCalcDegreeDaySumEndYear}
   {\theTUBlnThSiTenNiFiCalcDegreeDaySumCurrDay}
   {\theTUBlnThSiTenNiFiCalcDegreeDaySumCurrMon}
   {\theTUBlnThSiTenNiFiCalcDegreeDaySumCurrYear}

	%%http://texfaq.org/FAQ-repeat-num : package "ifthen"

%%https://tex.stackexchange.com/questions/587865/adding-a-number-to-command-parameter-and-outputting-it :
%% "\number\numexpr#1 + 20\relax%"

  %\TUBlnThSiTenNiFiIsLeapYear{\theTUBlnThSiTenNiFiCalcDegreeDaySumCurrYear}

  \TUBlnThSiTenNiFiSetVarVal{\TUBlnThSiTenNiFiNumFirstMonthDays}{
    \TUBlnThSiTenNiFiGregorianMonthDaysCount[
      \theTUBlnThSiTenNiFiCalcDegreeDaySumCurrMon]}

  \TUBlnThSiTenNiFiLog{Number of days in first month of time span:\TUBlnThSiTenNiFiNumFirstMonthDays}

  \CalcHeatngDegDays{}
  \TUBlnThSiTenNiFiCalcFirstHeatngDegMonths{}

  %%All days in 1st month processed/summed up.
  \setcounter{TUBlnThSiTenNiFiCalcDegreeDaySumCurrMon}{1}

  \TUBlnThSiTenNiFiCalcHeatngDegDayPromilleForFullYears
    {\TUBlnThSiTenNiFiCalcDegreeDaySumStrtYear}
    {\TUBlnThSiTenNiFiCalcDegreeDaySumStrtMon}
    {\TUBlnThSiTenNiFiCalcDegreeDaySumStrtDay}
    {\TUBlnThSiTenNiFiCalcDegreeDaySumEndYear}
    {\TUBlnThSiTenNiFiCalcDegreeDaySumEndMon}
    {\TUBlnThSiTenNiFiCalcDegreeDaySumEndDay}

%  \TUBlnThSiTenNiFiSetVarVal{\TUBlnThSiTenNiFiCalcHeatngDegreeDayNumFullYears}
%    {\the\numexpr\TUBlnThSiTenNiFiCalcHeatngDegreeDaySumLastFullYear-
%      \TUBlnThSiTenNiFiCalcHeatngDegreeDaySumFirstFullYear+1\relax}
%  \TUBlnThSiTenNiFiLog{number of full years:
%    \TUBlnThSiTenNiFiCalcHeatngDegreeDaySumLastFullYear - 
%    \TUBlnThSiTenNiFiCalcHeatngDegreeDaySumFirstFullYear+
%    1=
%    \TUBlnThSiTenNiFiCalcHeatngDegreeDayNumFullYears}

  \FPeval\TUBlnThSiTenNiFiCalcHeatngDegDayPromilleFullYears{
    %\value{TUBlnThSiTenNiFiCalcHeatngDegDayNumFullYears}
    \theTUBlnThSiTenNiFiCalcHeatngDegDayNumFullYears*1000}

  %%
  \ifthenelse
  {%\value{TUBlnThSiTenNiFiCalcHeatngDegDayNumFullYears}>0 \or
    %\(
    \TUBlnThSiTenNiFiCalcDegreeDaySumEndYear>\TUBlnThSiTenNiFiCalcDegreeDaySumStrtYear \and \TUBlnThSiTenNiFiCalcDegreeDaySumEndMon>1%\and \)
  }
  {
%    \TUBlnThSiTenNiFiCalcHeatngDegDayPromilleForLastMonths{#4}{#5}{#6}
%    \FPeval\TUBlnThSiTenNiFiCalcDegreeDaySumPromille
%       {%\TUBlnThSiTenNiFiCalcDegreeDaySumPromille+
%        %\TUBlnThSiTenNiFiGerHeatingDegDayCurrMonthPromilleFP
%        %\TUBlnThSiTenNiFiGerHeatingDegDayCurrMonthPromille
%        \TUBlnThSiTenNiFiGerHeatngDegDayLastMonthsPromille
%        %\TUBlnThSiTenNiFiCalcDegreeDaySumPromille
%        }
    \TUBlnThSiTenNiFiLog{End year(\TUBlnThSiTenNiFiCalcDegreeDaySumEndYear) >
start year(\TUBlnThSiTenNiFiCalcDegreeDaySumStrtYear) and end month(
\TUBlnThSiTenNiFiCalcDegreeDaySumEndMon) after January.}

    \TUBlnThSiTenNiFiSetVarVal{\TUBlnThSiTenNiFiGerHeatingDegDayCurrMonthDayCnt}
      {\TUBlnThSiTenNiFiGregorianMonthDaysCount[
        \TUBlnThSiTenNiFiCalcDegreeDaySumEndMon]
      }

    %\TUBlnThSiTenNiFiSetVarVal{\ooooo}{\TUBlnThSiTenNiFiCalcDegreeDaySumEndMon}
    \ifthenelse
    {
      \TUBlnThSiTenNiFiCalcDegreeDaySumEndDay=
        \TUBlnThSiTenNiFiGerHeatingDegDayCurrMonthDayCnt
    }
    {
      \TUBlnThSiTenNiFiSetVarVal{\ooooo}{\TUBlnThSiTenNiFiCalcDegreeDaySumEndMon}
    }
    {
      %%Not a full month
      \TUBlnThSiTenNiFiSetVarVal{\ooooo}{\the\numexpr\TUBlnThSiTenNiFiCalcDegreeDaySumEndMon-1\relax}
    }
    \TUBlnThSiTenNiFiCalcHeatngDegDayPromilleForFullMonths
      {%%TODO
       %\value{TUBlnThSiTenNiFiCalcHeatngDegreeDaySumCurrMon}
       %\theTUBlnThSiTenNiFiCalcHeatngDegreeDaySumCurrMon
       1
       }
      {%\TUBlnThSiTenNiFiCalcDegreeDaySumEndMon
       \ooooo}
    \FPset{\TUBlnThSiTenNiFiGerHeatngDegDayLastMonthsPromille}
      {\TUBlnThSiTenNiFiGerHeatngDegDayFullMonthsPromille}
  }
  {}
  %%No full year inbetween
  \ifthenelse
  {
    \TUBlnThSiTenNiFiCalcDegreeDaySumEndYear=
      \TUBlnThSiTenNiFiCalcDegreeDaySumStrtYear
  }
  {
    \TUBlnThSiTenNiFiLog{Start year=end year=>setting number of last months(already included in number of first months) to 0.}
    \FPset{\TUBlnThSiTenNiFiGerHeatngDegDayLastMonthsPromille}{0.0}
  }

  \TUBlnThSiTenNiFiCalcHeatngDegDayPromilleForLastDays{#4}{#5}{#6}
%%http://tex.stackexchange.com/questions/222732/increment-display-the-content-of-a-variable-section-number-under-latex
  %\endYear $\endYear+1 = \the\numexpr\endYear+1\relax$
  
			%%http://mirrors.ibiblio.org/CTAN/macros/generic/listofitems/listofitems-en.pdf :
			%%Accessing list items : "\>>list name<<[>>index<<]"

  \TUBlnThSiTenNiFiSetVarVal{%\TUBlnThSiTenNiFiCalcDegreeDaySumCurrMonDaysCount
    \TUBlnThSiTenNiFiGetGerHeatngDegDayPromilleMonthDaysCount}{
      \TUBlnThSiTenNiFiGregorianMonthDaysCount[
      \theTUBlnThSiTenNiFiCalcDegreeDaySumCurrMon]}

  \TUBlnThSiTenNiFiLog{1st full months of time span promille:
    \TUBlnThSiTenNiFiHeatngDegFirstFullMonthsPromille}

  \FPeval{\TUBlnThSiTenNiFiHeatngDegFirstDaysAndMonthsPromille}
   {\TUBlnThSiTenNiFiHeatngDegFirstDaysOfFirstMonthPromille +
    \TUBlnThSiTenNiFiHeatngDegFirstFullMonthsPromille}

  \FPeval\TUBlnThSiTenNiFiHeatngDegLastDaysAndMonthsPromille
    {\TUBlnThSiTenNiFiCalcHeatngDegreeDayLastDaysPromille+
     \TUBlnThSiTenNiFiGerHeatngDegDayLastMonthsPromille
    }

  \FPeval\TUBlnThSiTenNiFiHeatngDegOverallPromille
   {\TUBlnThSiTenNiFiHeatngDegFirstDaysAndMonthsPromille+
    %\TUBlnThSiTenNiFiHeatngDegFullYearsPromille+
    \TUBlnThSiTenNiFiCalcHeatngDegDayPromilleFullYears+
    \TUBlnThSiTenNiFiHeatngDegLastDaysAndMonthsPromille}
}
%}

%%%(c) Stefan Gebauer(Computer Science Master from TU Berlin)
%%@author Stefan Gebauer(TU Berlin matriculation number 361095)

%%from https://tex.stackexchange.com/questions/37094/what-is-the-recommended-way-to-assign-a-value-to-a-variable-and-retrieve-it-fors

%%Include guard(see http://en.wikipedia.org/wiki/Include_guard) to prevent
%% defining(via "\newcommand") a command more than once.
\ifdefined \TUBlnThSiTenNiFiSetVarVal
\else

\newcommand
%%Use unique prefix to avoid name clashes with other (La)Tex code (from other
%% packages).
%%TUBln=TU Berlin
%% ThSiTenNiFi=THree SIx TEN NIne FIve(matriculation number 36 10 95)
%%SetVarVal=SET VARiable VALue
 {\TUBlnThSiTenNiFiSetVarVal}[2]{
  \ifdefined #1%%Variable with name "#1" already exists/is already set.
    \renewcommand{#1}{#2}
  \else
    \newcommand{#1}{#2}
  \fi
}

\fi%%Include guard


%\newcounter{currYear}
%\newcounter{currMon}
%\newcounter{currDay}

%%The number of parameters to this function:
%% from:year,month,day,
%% to:year,month,day
%%@param #7:expected heating degree promille sum
\newcommand{\TUBlnThSiTenNiFiCalcHeatngDegDaySumTest}[8]
{
%	\setcounter{currYear}{#1}
%	\setcounter{currMon}{#2}
%	\setcounter{currDay}{#3}
	%\setVarVal{\endYearl}{#4}
%	\setVarVal{\ndyear}{#5}
	
%	\setVarVal{\endMon}{#5}
%	\setVarVal{\endDay}{#6}
	
%    \isLeapYear{\thecurrYear}
%	\ifthenelse{\theleapYear=1 \and
%		%%Februar
%		\thecurrMon=2}%%->29 Tage
%	{%\addtocounter{\currMonInDays}{1}
%		\thecurrYear is a leap year
%	}%%then
%	{\thecurrYear is not a leap year}%%else
	
%	%%Advance by 1 day
%	\ifthenelse{\thecurrDay < \currMonInDays }
%	{ \addtocounter{\currDay}{1} 
%		%\stepcounter{\currDay}
%	}%%then
%	{ \setVarVal{\newMon}{1} }%%else
%	
%	\ifthenelse{\newMon=1 \and \currMon=12 }
%	{ \setVarVal{\newYear}{1}
%		\setVarVal{\currMon}{1} }
%	{ }
%	\ifthenelse{\nextMon=1 \and \currMon < 12 }
%	{ \FPeval{\currMon}{\currMon+1} }
%	{ }
%	
%	\ifthenelse{\thecurrYear < \endYear}
%	{}
%	{}

  \TUBlnThSiTenNiFiSetVarVal{\TUBlnThSiTenNiFiCalcDegDaysSumNumNumExpctdDays}
    {#8}

  \TUBlnThSiTenNiFiCalcHeatngDegDaySum{#1}{#2}{#3}{#4}{#5}{#6}{#7}

  Calculating heating degree day sum from \TUBlnThSiTenNiFiCalcDegreeDaySumStrtDay.
  \TUBlnThSiTenNiFiCalcDegreeDaySumStrtMon(month).
  \TUBlnThSiTenNiFiCalcDegreeDaySumStrtYear\ to \TUBlnThSiTenNiFiCalcDegreeDaySumEndDay.
  \TUBlnThSiTenNiFiCalcDegreeDaySumEndMon(month).
  \TUBlnThSiTenNiFiCalcDegreeDaySumEndYear \newline

  Number of full years as integer:"%
    %\value{TUBlnThSiTenNiFiCalcHeatngDegDayNumFullYears}
    %%Use "the">>counter name<<to print representation
    \theTUBlnThSiTenNiFiCalcHeatngDegDayNumFullYears
    "
  %Heating degree days promille for \textbf{full years} of time span:
%  \TUBlnThSiTenNiFiCalcHeatngDegDayPromilleFullYears
  %\TUBlnThSiTenNiFiCalcDegreeDaySumPromille

  \begin{itemize}[noitemsep, topsep=0pt]
   \item Heating degree days promille for
   \color{blue}
   \textbf{first }
    %\TUBlnThSiTenNiFiCalcDegreeDaySumNumDaysTillMonEnd 
    \theTUBlnThSiTenNiFiCalcDegDaysSumNumFrstMonLastDays
   \textbf{ day(s)}
   \color{black}
   (=last days of first non-full month) of time span:
   \color{blue}
     \TUBlnThSiTenNiFiHeatngDegFirstDaysOfFirstMonthPromille
   \color{black}
   \item 
      \ifthenelse
      {\TUBlnThSiTenNiFiCalcDegreeDaySumEndDay<
        \TUBlnThSiTenNiFiCalcHeatngDegNumEndMonthDays
      }%%"
      {
        Last day(\TUBlnThSiTenNiFiCalcDegreeDaySumEndDay) of last month of time span is not the last day of this month(\TUBlnThSiTenNiFiCalcHeatngDegNumEndMonthDays).
      }
      {}
    \item Heating degree days promille for
    \color{blue}
    \textbf{first}
      \theTUBlnThSiTenNiFiCalcGerHeatngDegDaysSumNumFirstFullMons\ 
    \textbf{full month(s)}
    \color{black}\
    of non-full years %in non-full years
   of time span:
   \color{blue}
    \TUBlnThSiTenNiFiHeatngDegFirstFullMonthsPromille
   \color{black}

  \item Heating degree day promille
  \color{blue}
  \textbf{last} \TUBlnThSiTenNiFiCalcGerHeatngDegDaysSumNumFullEndMons\
  \textbf{full month(s)}
  \color{black}
  of non-full years of time span:
  \color{blue}
    \TUBlnThSiTenNiFiGerHeatngDegDayLastMonthsPromille
  \color{black}

  \item Heating degree day promille for
  \color{blue}
  \textbf{last} \TUBlnThSiTenNiFiCalcDegreeDaySumEndDay \ \textbf{day(s)}
  \color{black}
  (of last non-full month) of time span:
  \color{blue}
    \TUBlnThSiTenNiFiCalcHeatngDegreeDayLastDaysPromille
  \color{black}

  %current year: \theTUBlnThSiTenNiFiCalcDegreeDaySumCurrYear
  \TUBlnThSiTenNiFiIsLeapYear{\theTUBlnThSiTenNiFiCalcDegreeDaySumCurrYear}
  \ifthenelse
  {\theTUBlnThSiTenNiFiLeapYear=1 \and
    %%Februar
    \theTUBlnThSiTenNiFiCalcDegreeDaySumCurrMon=2
  }%%->29 Tage
  {%\addtocounter{%\TUBlnThSiTenNiFiCalcDegreeDaySumCurrMonDaysCount
    % \TUBlnThSiTenNiFiGetGerHeatngDegDayPromilleMonthDaysCount}{1}
    \theTUBlnThSiTenNiFiCalcDegreeDaySumCurrYear is a leap year and February
  }%%then
  {\theTUBlnThSiTenNiFiCalcDegreeDaySumCurrYear is not:a leap year and February}%%else

 \item 
  \color{blue}
  \TUBlnThSiTenNiFiHeatngDegFirstDaysOfFirstMonthPromille(\textbf{first days}
  \color{black}
  of first non-full month of time span)
  + \color{blue}
  \TUBlnThSiTenNiFiHeatngDegFirstFullMonthsPromille(\textbf{first full months}
  \color{black}
  of first non-full year of time span)=%%
  \TUBlnThSiTenNiFiHeatngDegFirstDaysAndMonthsPromille %\TUBlnThSiTenNiFiCalcDegreeDaySumPromille

 \item 
  \color{blue}
  \TUBlnThSiTenNiFiCalcHeatngDegreeDayLastDaysPromille(\textbf{last days}
  \color{black}
  of last non-full month of time span) +
  \color{blue}
    \TUBlnThSiTenNiFiGerHeatngDegDayLastMonthsPromille(\textbf{last full months}
  \color{black}
  of last non-full year of time span)
  =\TUBlnThSiTenNiFiHeatngDegLastDaysAndMonthsPromille

 \item 
  \textbf{Overall} promille:\newline
  \color{blue}
  \TUBlnThSiTenNiFiHeatngDegFirstDaysAndMonthsPromille(first days and months
  \color{black}
  promille of time span)+\newline
  \color{blue}
  %\TUBlnThSiTenNiFiHeatngDegFullYearsPromille
  \TUBlnThSiTenNiFiCalcHeatngDegDayPromilleFullYears
  (full years
  \color{black}
  promille of time span)+\newline
  \color{blue}
  \TUBlnThSiTenNiFiHeatngDegLastDaysAndMonthsPromille(last days and months
  \color{black}
  promille of time span)=\newline
   \TUBlnThSiTenNiFiHeatngDegOverallPromille\newline
 \end{itemize}

  \FPeval\TUBlnThSiTenNiFiHeatngDegOverallPromilleError{
    \TUBlnThSiTenNiFiHeatngDegOverallPromille/
    \TUBlnThSiTenNiFiCalcDegDaySumExpctdRslt}
  Deviation from expected value(result/expected result):
   \TUBlnThSiTenNiFiHeatngDegOverallPromilleError
  \newline
  \TUBlnThSiTenNiFiSetVarVal{\TUBlnThSiTenNiFiHeatngDegOverallPromilleOutOftRng}{0}
  \FPifgt{\TUBlnThSiTenNiFiHeatngDegOverallPromilleError}{1.01}
    %%"then" branch?
    \TUBlnThSiTenNiFiSetVarVal{\TUBlnThSiTenNiFiHeatngDegOverallPromilleOutOftRng}{1}
  \else
  \fi
  \FPifgt{0.99}{\TUBlnThSiTenNiFiHeatngDegOverallPromilleError}
    %%"then" branch?
    \TUBlnThSiTenNiFiSetVarVal{\TUBlnThSiTenNiFiHeatngDegOverallPromilleOutOftRng}{1}
  \else
  \fi
  \ifthenelse
  {
    \TUBlnThSiTenNiFiHeatngDegOverallPromilleOutOftRng=1
  }%%"if" branch
  {%%"then" branch
    \color{red}Result does not match expected result
      "\TUBlnThSiTenNiFiCalcDegDaySumExpctdRslt"+-1\%.
     \color{black}
  }%%"then" branch
  {
    \color{green}Result matches expected result
      „%\textit{
       \TUBlnThSiTenNiFiCalcDegDaySumExpctdRslt%}
      “ +-1\%.\color{black}
  }%%"else" branch

  date difference: %\TUBlnThSiTenNiFiSubtractTwoDatesYearDiff
   %years
   \theTUBlnThSiTenNiFiCalcDegDaysSumNumDays\ days

  \ifthenelse
  {
    \theTUBlnThSiTenNiFiCalcDegDaysSumNumDays =
    \TUBlnThSiTenNiFiCalcDegDaysSumNumNumExpctdDays
  }%%"if" branch
  {%%"then" branch
    \color{green}Number of days matches expected result.\color{black}
  }%%"then" branch
  {
    \color{red}Number of days does not match expected result
     "\TUBlnThSiTenNiFiCalcDegDaysSumNumNumExpctdDays".
     \color{black}
  }%%"else" branch
}

\title{„German heating degree day sum% function
  “ test}
\author{Stefan Gebauer(TU Berlin matriculation number 361095)}

%% \calcDegreeDaySum

\newcommand\testTwntyFourNovTwoThsdTwntyThrTillFrstMayTwoThsdTwntyFour{
 \TUBlnThSiTenNiFiCalcHeatngDegDaySumTest
  {2023}{11}{24}{2024}{5}{1}{
  %%Expected promille sum and expected days count:
  %%  28       7(heating degree promille per day in November)*7 days=120/30*7
  %%+160     +31(heating degree promille per month in December)
  %%=188     =38(first days and first months of time span)
%
  %% 170      31(heating degree promille per month in January)
  %%+150     +29(heating degree promille per month for 29 days in February)
  %%+130     +31(heating degree promille per month in March)
  %%+ 80     +30(heating degree promille per month in April)
  %%+  1,29   +1(heating degree promille for 1 day in May)
  %%=531,29 =122(last days and last months of time span)
%
  %% 188      38(first days and first months)
  %%+531,29 +122(last days and last months)
  %%=719.29 =160=first days and first months plus last days and last months
  719.29}{160}
}
\newcommand\testXXIVNovMMXXIIItillIMayMMXXIV{
 \testTwntyFourNovTwoThsdTwntyThrTillFrstMayTwoThsdTwntyFour
}
\begin{document}
\maketitle

\color{orange}
Use "debug" option for "fp" package if "fp" calculation errors exist to see the cause.
\newline

All days are included, so the 01.May-02.May count as 2 days, 01.May-02.May as
 only 1 day.

%Number of days for each month in Gregorian calendar:
% \showitems\TUBlnThSiTenNiFiGregorianMonthDaysCount .

%%https://mirrors.ibiblio.org/CTAN/macros/generic/listofitems/listofitems-en.pdf :
%% "\readlist\phrase{One phrase to test.}
%%  \foreachitem\word\in\phrase{List item number \wordcnt{}: \word\par}"
%\foreachitem\GregorianMonthDaysCount\in\TUBlnThSiTenNiFiGregorianMonthDaysCount⟩{ \GregorianMonthDaysCountcnt{}: \GregorianMonthDaysCount\par }

%%Following source code adapted from:
%% https://tex.stackexchange.com/questions/7590/how-to-programmatically-make-tabular-rows-using-whiledo

%%Create table cells with Gregorian month index:
\newcounter{MonthIdx}\newcounter{AddMonthNumsAsTabCellsIdx}

\makeatletter
\newtoks\@MonthNumsAsTabCells
\newcommand\addMonthNumsAsTabCells[1]{
 \@MonthNumsAsTabCells\expandafter{\the\@MonthNumsAsTabCells#1} }
\newcommand*\resetMonthNumsAsTabCells{\@MonthNumsAsTabCells{}}
\newcommand*\printMonthNumsAsTabCells{\the\@MonthNumsAsTabCells}
\makeatother

\resetMonthNumsAsTabCells
\loop\ifnum\theAddMonthNumsAsTabCellsIdx<
 \TUBlnThSiTenNiFiGregorianMonthDaysCountlen\relax
  \stepcounter{AddMonthNumsAsTabCellsIdx}
  \addMonthNumsAsTabCells{\stepcounter{MonthIdx} \theMonthIdx & 
   }
\repeat
\addMonthNumsAsTabCells{\\%\hline
 }
%%Create table cells with Gregorian month index end.

%%Following source code adapted from:
%% https://tex.stackexchange.com/questions/7590/how-to-programmatically-make-tabular-rows-using-whiledo

%%Create table cells from Gregorian month days count:
\makeatletter
\newtoks\@MonthDaysCntAsTabCells
\newcommand\addMonthDaysCntAsTabCells[1]{
 \@MonthDaysCntAsTabCells\expandafter{\the\@MonthDaysCntAsTabCells#1} }
\newcommand*\resetMonthDaysCntAsTabCells{\@MonthDaysCntAsTabCells{}}
\newcommand*\printMonthDaysCntAsTabCells{\the\@MonthDaysCntAsTabCells}
\makeatother

\setcounter{MonthIdx}{0}
\resetMonthDaysCntAsTabCells
\newcounter{monthDaysCntAsTabCellsCtr}
\newcounter{MonthDaysCntAsTabCellsLoopCtr}
\loop\ifnum\theMonthDaysCntAsTabCellsLoopCtr<
 \TUBlnThSiTenNiFiGregorianMonthDaysCountlen\relax
  \stepcounter{MonthDaysCntAsTabCellsLoopCtr}
   \addMonthDaysCntAsTabCells{
      \stepcounter{monthDaysCntAsTabCellsCtr}  
      %\themonthDaysCntAsTabCellsCtr \ 
      \TUBlnThSiTenNiFiGregorianMonthDaysCount[\themonthDaysCntAsTabCellsCtr] & 
   }%
\repeat
\addMonthDaysCntAsTabCells{\\%\hline
 }
%%Create table cells from Gregorian month days end

%%Following source code adapted from:
%% https://tex.stackexchange.com/questions/7590/how-to-programmatically-make-tabular-rows-using-whiledo

%%Create table cells from Gregorian month days count:
\makeatletter
\newtoks\@GerHeatngDegPromillePerDayAsTabCells
\newcommand\addGerHeatngDegPromillePerDayAsTabCells[1]{
 \@GerHeatngDegPromillePerDayAsTabCells\expandafter{\the\@GerHeatngDegPromillePerDayAsTabCells#1} }
\newcommand*\resetGerHeatngDegPromillePerDayAsTabCells{\@GerHeatngDegPromillePerDayAsTabCells{}}
\newcommand*\printGerHeatngDegPromillePerDayAsTabCells{\the\@GerHeatngDegPromillePerDayAsTabCells}
\makeatother

\resetGerHeatngDegPromillePerDayAsTabCells
\newcounter{GerHeatngDegPromillePerDayAsTabCellsCtr}
\newcounter{GerHeatngDegPromillePerDayAsTabCellsLoopCtr}
\loop\ifnum\theGerHeatngDegPromillePerDayAsTabCellsLoopCtr<
 \TUBlnThSiTenNiFiGerHeatingDegDayPromillePerDayTwoDecPlaceslen\relax
  \stepcounter{GerHeatngDegPromillePerDayAsTabCellsLoopCtr}
   \addGerHeatngDegPromillePerDayAsTabCells{
      \stepcounter{GerHeatngDegPromillePerDayAsTabCellsCtr}  
      %\themonthDaysCntAsTabCellsCtr \ 
     \TUBlnThSiTenNiFiGerHeatingDegDayPromillePerDayTwoDecPlaces[
      \theGerHeatngDegPromillePerDayAsTabCellsCtr] & 
   }%
\repeat
\addGerHeatngDegPromillePerDayAsTabCells{\\%\hline
 }
%%Create table cells from Gregorian month days end

%%Following source code adapted from:
%% https://tex.stackexchange.com/questions/7590/how-to-programmatically-make-tabular-rows-using-whiledo

%%Create table cells from Gregorian month days count:
\makeatletter
\newtoks\@GerHeatngDegPromillePerMonthAsTabCells
\newcommand\addGerHeatngDegPromillePerMonthAsTabCells[1]{
 \@GerHeatngDegPromillePerMonthAsTabCells\expandafter{\the\@GerHeatngDegPromillePerMonthAsTabCells#1} }
\newcommand*\resetGerHeatngDegPromillePerMonthAsTabCells{\@GerHeatngDegPromillePerMonthAsTabCells{}}
\newcommand*\printGerHeatngDegPromillePerMonthAsTabCells{\the\@GerHeatngDegPromillePerMonthAsTabCells}
\makeatother

\resetGerHeatngDegPromillePerMonthAsTabCells
\newcounter{GerHeatngDegPromillePerMonthAsTabCellsCtr}
\newcounter{GerHeatngDegPromillePerMonthAsTabCellsLoopCtr}
\loop\ifnum\theGerHeatngDegPromillePerMonthAsTabCellsLoopCtr<
 \TUBlnThSiTenNiFiGerHeatingDegDayPromillePerMonlen\relax
  \stepcounter{GerHeatngDegPromillePerMonthAsTabCellsLoopCtr}
   \addGerHeatngDegPromillePerMonthAsTabCells{
      \stepcounter{GerHeatngDegPromillePerMonthAsTabCellsCtr}  
      %\themonthDaysCntAsTabCellsCtr \ 
      \TUBlnThSiTenNiFiGerHeatingDegDayPromillePerMon[\theGerHeatngDegPromillePerMonthAsTabCellsCtr] & 
   }%
\repeat
\addGerHeatngDegPromillePerMonthAsTabCells{\\%\hline
 }
%%Create table cells from Gregorian month days end


\begin{longtable}{c|c|c|c|c|c|c|c|c|c|c|c|c|c}
 \hline
 month index: & \printMonthNumsAsTabCells
 \hline
 \# month days: & \printMonthDaysCntAsTabCells
 \hline
 promille/mon & \printGerHeatngDegPromillePerMonthAsTabCells
 \hline
 promille/day & \printGerHeatngDegPromillePerDayAsTabCells
 \hline
\end{longtable}

%Heating degree promille per day for each month:
%\showitems\TUBlnThSiTenNiFiGerHeatingDegDayPromillePerDayTwoDecPlaces .
%%
%Heating degree promille per month for each month:
%\showitems\TUBlnThSiTenNiFiGerHeatingDegDayPromillePerMon

%
 \TUBlnThSiTenNiFiCalcHeatngDegDaySumTest{2023}{10}{03}{2023}{10}{31}{%%28*2,581}
  72.268}{28}
%\TUBlnThSiTenNiFiCalcDegreeDaySum{2004}{10}{30}{2006}{11}{30}{2242.48}
\color{black}%\newline
%\TUBlnThSiTenNiFiCalcHeatngDegDaySumTest{2023}{10}{31}{2023}{11}{23}{
%  %%Expected promille sum: 120(heating degree promille per day in November)*23
%  %% days=120/30*23=92,
%  92.0}{23}

 %\testTwntyFourNovTwoThsdTwntyThrTillFrstMayTwoThsdTwntyFour
 \testXXIVNovMMXXIIItillIMayMMXXIV
%\calcDegreeDaySum{1999}
%\calcDegreeDaySum{2000}
%\calcDegreeDaySum{2024}  
\end{document}